\documentclass[11pt]{article}
\usepackage[paper=letterpaper,margin=0.5in,footskip=0.25in]{geometry}
\usepackage{amsmath}
\usepackage{amssymb}
\usepackage{titlesec}
\usepackage{enumitem}
\usepackage{graphicx}
\usepackage{href-ul}
\usepackage{xcolor}
\usepackage{siunitx}

\hypersetup{
	colorlinks   = true,    % colored links instead of boxes
	urlcolor     = blue,    % color for external hyperlinks
	linkcolor    = blue,    % color for internal links
}

\makeatletter
\def\@maketitle{
	\begin{center}
		\LARGE\@title\smallskip\par
		\normalsize\@author\medskip\par
		\href{https://www.linkedin.com/in/erk-sampat-500331245/}{LinkedIn}\hfill%
		\href{esampat@ucsb.edu}{esampat@ucsb.edu}\hfill%
		\href{https://sar-mango.github.io/}{Personal Website}\hfill%
		\href{https://github.com/SAR-mango}{GitHub}\hfill
	\end{center}
}
\makeatother

\titleformat{\section}{}{}{0em}{\large\textbf}
\titleformat{\subsection}{}{}{0em}{\textbf}
\titlespacing*{\section}{0em}{0.75em}{0em}
\titlespacing*{\subsection}{0em}{0.5em}{\lineskip}

\title{\textbf{Erk Sampat}}
\author{University of California, Santa Barbara -- B.S., Electrical Engineering (June 2026)}

\pagenumbering{gobble}
\begin{document}
	\maketitle
	\section*{SKILLS}
	\hrule\medskip
	\begin{minipage}{0.5\textwidth}
		\begin{itemize}[itemsep=0em]
			\item Analog integrated circuit design
			\item Power electronics; DC-DC converter design
			\item Embedded systems (register-level STM32)
			\item Digital logic design
			\item Circuit simulation (LTSpice)
			\item Hardware and software debugging
			\item Transmission line and antenna theory
		\end{itemize}
	\end{minipage}%
	\begin{minipage}{0.5\textwidth}
		\begin{itemize}[itemsep=0em]
			\item C and C++; MATLAB; \texttt{git}
			\item Data structures and algorithms
			\item Object-oriented programming
			\item Technical documentation; \LaTeX
			\item High-density PCB design (KiCAD, Altium)
			\item Electronics lab instrumentation
			\item SMD rework
		\end{itemize}
	\end{minipage}
	\section*{PROJECTS}
	\hrule\smallskip
	\subsection{\href{https://github.com/SAR-mango/SinESC}{SinESC}}
	Highly efficient brushless motor controller for drones. Supports sensorless field-oriented control for maximal power efficiency, resulting in increased flight time and smoother operation.
	\subsection{Delta-Sigma Audio Amplifier}
	Circuit-level implementation of a delta-sigma data converter and full-bridge GaN output stage. Provides \qty{20}{\watt} of output power with efficiency over 90\%.
	\subsection{RC Drones and Planes}
	Built and flew remote-controlled drones and planes -- primarily first-person-view drones, glider-style airplanes, and nitromethane-powered airplanes. Set up autopilot and Iridium satellite communications for an airplane designed to detect forest fires. Won \href{https://www.hackster.io/team-sol/tinyml-aerial-forest-fire-detection-78ec6b}{first place} in tinyML Vision Challenge.
	\subsection{\href{https://github.com/SAR-mango/Universal-Laser-Driver}{Universal Laser Driver}}
	Low-power laser driver with ultra-wide input and output voltage range. Also built laser pointers of various wavelengths using the Universal Laser Driver.
	\subsection{Solid-State Tesla Coil}
	Converts line voltage to several hundred kilovolts, generating foot-long electrical arcs. Used to demonstrate high voltage and electromagnetic induction.
	\section*{EXPERIENCE}
	\hrule\smallskip
	\subsection{Undergraduate Researcher -- Biomimetic Circuits and Nanosystems Group (July 2025--present)}
		Designing a metalens (composed of TiO\textsubscript{2} nanofins) to be fabricated on top of a CMOS SPAD sensor. Applications include probabilistic computing and biosensors. Advised by Prof. Luke Theogarajan; intend to publish in 2026.
	\subsection{Power Electronics Intern -- Astranis Space Technologies (June--September 2024)}
	\begin{itemize}[itemsep=0em]
		\item Designed, tested, and documented a radiation testing board for 16 different high-voltage diodes. Implemented high-voltage biasing, clamping, and transient fault-detection circuitry. Hardware architecture to be reused for future radiation tests.
		\item Designed a four-channel \qty{80}{\volt}/\qty{60}{\ampere} high-side GaN load stepping board for testing flight hardware. Implemented power stage, including linear current ramp soft-starting regime. Also realized under-voltage lockout, over-current protection, and over-voltage protection. Voltage/current telemetry made accessible to the user.
		\item Wrote a script using Google's \texttt{openhtf} to automate radiation tests for op-amps.
		\item Assisted in debugging a faulty flyback converter in battery management system.
	\end{itemize}
	\subsection{Student Worker -- UCSB Department of Physics (April 2023--June 2024)}
	\begin{center}
		\begin{minipage}{0.5\textwidth}
			\begin{itemize}[itemsep=0em]
				\item Set up labs for all undergraduate physics courses
				\item Managed inventory; repaired broken test equipment
			\end{itemize}
		\end{minipage}%
		\begin{minipage}{0.5\textwidth}
			\begin{itemize}[itemsep=0em]
				\item Documented lab procedures
				\item Designed electronics projects for lab curriculum
			\end{itemize}
		\end{minipage}
	\end{center}
\end{document}
